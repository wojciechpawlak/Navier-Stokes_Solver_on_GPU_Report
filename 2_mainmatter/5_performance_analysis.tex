\chapter{Performance Analysis}

Synchronization between work-items is possible only within workgroups using barriers and memory fences. It is not possible to synchronize workitems that are in different workgroups. This would hinder data parallelism.

Memory management is explicit, so a programmer moves data from host to global memory of device first using OpenCL constructs. Then it can be moved to local memory. To read the results on the host the process has to be reverted.

Workgroup size for a given algorithm should be selecten to be an even multiple of the width of the hardware scheduling units.

\section{Computed results}
Theoretic bandwidth vs. Effective bandwidth

Trap of premature optimization

Visual Profiler Recommendations

Bandwidth

Data Transfer Between Host and Device

PCIe

coalescing global memory accesses

compute capability 2.x

Branching and Divergence
Avoid different execution paths within the same warp.

vector operations, vector variables

built-in math functions, fast optimized library

Syncrhonization on Global and Local barriers

Synchronization between kernels in the queue. All the kernels are waiting for another to finish to work on nits results.
One device - one queue.

CPU code in VIsual studio 2010 basic release build optimizations on both CPU execution and CPU control parts of GPU execution

Asynchronous reading of memory $CL_FALSE$ instead of $CL_TRUE$

allocating memory withou copyting from host pointer

No print options in the loop

Workgroup size

Size of the grid

Kernels are executed asynchronously
enquing memory resources asynchronously
using events to track the execution status

